  \subsection{Network}
  \begin{frame}
    \frametitle{Aims}
      \begin{itemize}
        \item Interface transport layer,\pause
	\item Host addressing,\pause
        \item End-to-end packet transmission (data link? Connectionless? Switch? Router?),\pause
        \item Routing, load balancing
      \end{itemize}
  \end{frame}
  \subsubsection{IP addressing}
  \begin{frame}
    \frametitle{Concepts}
      \begin{itemize}
        \item IP addressing fundamentals,\pause
        \item Classfull IP addressing,\pause
        \item Subnet and VLSM (Variable length subnet masks),\pause
        \item CIDR (Classless inter-domain routing).
      \end{itemize}
  \end{frame}

  \begin{frame}
    \frametitle{IP addressing fundamentals}
    \begin{block}{IP address}
      \begin{figure}
        \centering
        \begin{tabular}{|c|c|}
          \multicolumn{2}{c}{32 bits (4x4 bytes)} \\ \hline
           \multicolumn{2}{|c|}{\color{brown}mask} \\ \hline
          \color{ForestGreen}Networks part & \color{blue}Host part \\ \hline
        \end{tabular}
        \caption{IP address parts}
        \label{fig:inside_ip_address}
      \end{figure}
    \end{block}
  \end{frame}

  \begin{frame}
    \frametitle{IP addressing fundamentals}
    \begin{block}{Masks}
      \begin{itemize}
        \item Separates {\color{ForestGreen}network} and {\color{blue}host} bits,\pause
        \item MSB \textbf{always} are ones and then zeros! 255.254.255.0 is not possible,\pause
        \item Indicates how many bits are used for the {\color{ForestGreen}network} part:
        \begin{itemize}
          \item A 8-bit {\color{brown}mask} leaves 24 bits for the {\color{blue}hosts},
          \item A 16-bit {\color{brown}mask} leaves 16 bits for the {\color{blue}hosts},
          \item A 24-bit {\color{brown}mask} leaves 8 bits for the {\color{blue}hosts},
          \item A N-bit {\color{brown}mask} leaves 32-N bits for the {\color{blue}hosts}.
        \end{itemize}\pause
        \item Two different {\color{brown}masks} (differences seen further):
        \begin{itemize}
          \item Network {\color{brown}mask},
          \item Subnet {\color{brown}mask}.
        \end{itemize}
      \end{itemize}
    \end{block}
  \end{frame}
  \begin{frame}
    \frametitle{IP addressing fundamentals}
    \begin{block}{IP address}
      \begin{figure}
        \centering
        \begin{tabular}{|c|c|}
          \multicolumn{2}{c}{32 bits (4x4 bytes)} \\ \hline
          \uncover<2->{\color{brown}ones mask} & \uncover<2->{\color{fuchsia}zeros mask} \\ \hline
          \color{ForestGreen}Networks part & \color{blue}Host part \\ \hline
        \end{tabular}
        \caption{IP address parts and {\color{brown}mask}}
        \label{fig:inside_ip_address_mask}
      \end{figure}
    \end{block}
  \end{frame}

  \begin{frame}
    \frametitle{IP addressing fundamentals}
    \begin{block}{Is that an address?}
      \begin{itemize}
        \item Network address,\pause
        \item Hosts,\pause
        \item Broadcast address.\pause
      \end{itemize}
    \end{block}
    \begin{block}{Within the same network}
      \begin{itemize}
        \item All addresses have the same {\color{ForestGreen}network} bits,\pause
        \item Network address has zeros for {\color{blue}host} bits: {\color{ForestGreen}x.x.x}.{\color{blue}0*},\pause
        \item All {\color{blue}hosts} have different {\color{blue}host} bits: {\color{ForestGreen}x.x.x}.{\color{blue}[0-1]*},\pause
        \item Broadcast address has ones for {\color{blue}host} bits: {\color{ForestGreen}x.x.x}.{\color{blue}1*}.
      \end{itemize}
    \end{block}
  \end{frame}

  \begin{frame}
    \frametitle{IP addressing fundamentals}
    \begin{figure}
        \centering
      \begin{tabular}{|r|cccc|}
        \hline
        Mask {\color{brown}/24} & {\color{brown}255} & {\color{brown}255} & {\color{brown}255} & {\color{fuchsia}0} \\
        254 {\color{blue}hosts}& {\color{brown}11111111} & {\color{brown}11111111} & {\color{brown}11111111} & {\color{fuchsia}00000000} \\ \hline
        \multirow{2}{*}{Network address} & \color{ForestGreen}192 & \color{ForestGreen}168 & \color{ForestGreen}1 & \color{blue}0 \\
        & \color{ForestGreen}11000000 & \color{ForestGreen}10101000 & \color{ForestGreen}00000001 & \color{blue}00000000 \\ \hline
        \multirow{2}{*}{First host} & \color{ForestGreen}192 & \color{ForestGreen}168 & \color{ForestGreen}1 & \color{blue}1 \\
        & \color{ForestGreen}11000000 & \color{ForestGreen}10101000 & \color{ForestGreen}00000001 & \color{blue}00000001 \\ \hline
        \multirow{2}{*}{Last host} & \color{ForestGreen}192 & \color{ForestGreen}168 & \color{ForestGreen}1 & \color{blue}254 \\
        & \color{ForestGreen}11000000 & \color{ForestGreen}10101000 & \color{ForestGreen}00000001 & \color{blue}11111110 \\ \hline
        \multirow{2}{*}{Broadcast address} & \color{ForestGreen}192 & \color{ForestGreen}168 & \color{ForestGreen}1 & \color{blue}255 \\
        & \color{ForestGreen}11000000 & \color{ForestGreen}10101000 & \color{ForestGreen}00000001 & \color{blue}11111111 \\ \hline
      \end{tabular}
      \caption{IP address example 1}
    \end{figure}
  \end{frame}

  \begin{frame}
    \frametitle{IP addressing fundamentals}
    \begin{figure}
        \centering
      \begin{tabular}{|r|cccc|}
        \hline
        Mask {\color{brown}/16} & {\color{brown}255} & {\color{brown}255} & {\color{fuchsia}0} & {\color{fuchsia}0} \\
        65.534 {\color{blue}hosts} & {\color{brown}11111111} & {\color{brown}11111111} & {\color{fuchsia}00000000} & {\color{fuchsia}00000000} \\ \hline
        \multirow{2}{*}{Network address} & \color{ForestGreen}172 & \color{ForestGreen}64 & \color{blue}0 & \color{blue}0 \\
        & \color{ForestGreen}10101100 & \color{ForestGreen}01000000 & \color{blue}00000000 & \color{blue}00000000 \\ \hline
        \multirow{2}{*}{First host} & \color{ForestGreen}172 & \color{ForestGreen}64 & \color{blue}0 & \color{blue}1 \\
        & \color{ForestGreen}10101100 & \color{ForestGreen}01000000 & \color{blue}00000000 & \color{blue}00000001 \\ \hline
        \multirow{2}{*}{Last host} & \color{ForestGreen}172 & \color{ForestGreen}64 & \color{blue}255 & \color{blue}254 \\
        & \color{ForestGreen}10101100 & \color{ForestGreen}01000000 & \color{blue}11111111 & \color{blue}11111110 \\ \hline
        \multirow{2}{*}{Broadcast address} & \color{ForestGreen}172 & \color{ForestGreen}64 & \color{blue}255 & \color{blue}255 \\
        & \color{ForestGreen}10101100 & \color{ForestGreen}01000000 & \color{blue}11111111 & \color{blue}11111111 \\ \hline
      \end{tabular}
      \caption{IP address example 2}
    \end{figure}
  \end{frame}

  \begin{frame}
    \frametitle{IP addressing fundamentals}
    \begin{block}{\textbf{Formula}: how many {\color{blue}hosts} with a N-bit mask?}
      $2^{32-N}-2$\pause, the $-2$ moves out network and broadcast addresses which are not {\color{blue}hosts}.\pause
      \begin{itemize}
        \item 24-bit {\color{brown}mask}: $2^{32-24}-2 = 2^{8}-2 = 254$ {\color{blue}hosts} \pause
        \item 16-bit {\color{brown}mask}: $2^{32-16}-2 = 2^{16}-2 = 65.534$ {\color{blue}hosts} \pause
        \item 8-bit {\color{brown}mask}: $2^{32-8}-2 = 2^{24}-2 = 16.777.214$ {\color{blue}hosts}
      \end{itemize}
    \end{block}
  \end{frame}

  \begin{frame}
    \frametitle{IP addressing fundamentals}
    \begin{block}{Public addresses}
      \begin{itemize}
        \item Most of IP addresses \pause
        \item Registered ISP and large organizations inherit blocks of public addresses from IANA\footnote{Internet Assigned Numbers Authority} \pause
        \item Usage of not registered public addresses is forbidden.
      \end{itemize}
    \end{block}
    \begin{block}{Private addresses}
      \begin{itemize}
        \item Privates addresses are A, B and C classes (note all, see after)\pause
        \item No registration needed \pause
        \item Not routed across the Internet \pause
        \item Proxy, NAT and private addresses solved IPv4 shortage.
      \end{itemize}
    \end{block}
  \end{frame}

  \begin{frame}
    \frametitle{Classful IP Addressing}
    \begin{figure}
      \centering
      \resizebox{11.5cm}{!} {
	\begin{tabular}{|r||c|c|c|}
	  \hline
	  Class & A & B & C \\ \hline \hline
	  First octet & {\color{ForestGreen}1} - {\color{ForestGreen}126} & {\color{ForestGreen}128} - {\color{ForestGreen}191} & {\color{ForestGreen}192} - {\color{ForestGreen}223} \\ \hline
	  First octet 0b& {\color{ForestGreen}0*} & {\color{ForestGreen}10*} & {\color{ForestGreen}110*} \\ \hline
	  \multirow{2}{*}{\color{brown}Network mask} & {\color{brown}255}.{\color{fuchsia}0.0.0} & {\color{brown}255.255}.{\color{fuchsia}0.0} & {\color{brown}255.255.255}.{\color{fuchsia}0}\\
	  & {\color{brown}/8} & {\color{brown}/16} & {\color{brown}/24} \\ \hline
	  \multirow{2}{*}{IP addresses range} & {\color{ForestGreen}1}.{\color{blue}0.0.0} & {\color{ForestGreen}128.0}.{\color{blue}0.0} & {\color{ForestGreen}192.0.0}.{\color{blue}0}\\
	  & {\color{ForestGreen}126}.{\color{blue}0.0.0} & {\color{ForestGreen}191.255}.{\color{blue}0.0} & {\color{ForestGreen}223.255.255}.{\color{blue}0} \\ \hline
	  \multirow{2}{*}{Private range}
	  & {\color{ForestGreen}10}.{\color{blue}0.0.0} 	    & {\color{ForestGreen}176.16}.{\color{blue}0.0}     & {\color{ForestGreen}192.168.0}.{\color{blue}0}\\
	  & {\color{ForestGreen}10}.{\color{blue}255.255.255} & {\color{ForestGreen}176.31}.{\color{blue}255.255} & {\color{ForestGreen}192.168.255}.{\color{blue}0} \\ \hline
	  Number of {\color{blue}hosts} & 16.777.214 & 65.534 & 254 \\ \hline
	\end{tabular}
      }
      \caption{Three main classes}
    \end{figure}
    Where did {\color{ForestGreen}127}.{\color{blue}0.0.0}{\color{brown}/8} go ?!
  \end{frame}

  \begin{frame}
    \frametitle{Classful IP Addressing}
    \begin{block}{Class D}
      \begin{itemize}
	\item First octet: {\color{ForestGreen}224} - {\color{ForestGreen}239} \pause
	\item First octet pattern: {\color{ForestGreen}1110*} \pause
	\item Theses IP addresses are multicast addresses.\pause
      \end{itemize}
    \end{block}
    \begin{block}{Class E}
      \begin{itemize}
	\item Everything left \pause
	\item Experimental class.
      \end{itemize}
    \end{block}
  \end{frame}

  \begin{frame}
    \frametitle{Classful IP Addressing}
    \begin{block}{Reserved addresses}
      \begin{itemize}
	\item 0.0.0.0 used in routing (seen further) \pause
	\item {\color{ForestGreen}127}.{\color{blue}0.0.0}{\color{brown}/8}: loopback addresses ({\color{ForestGreen}127}.{\color{blue}0.0.1} - {\color{ForestGreen}127}.{\color{blue}255.255.254}).
      \end{itemize}
    \end{block}
  \end{frame}

  \begin{frame}
    \frametitle{Classful IP Addressing}
    \begin{itemize}
      \item Class A (16 m-addresses) and B (65 k-adresses) are too large! \pause
      \item Class C (254 addresses) is manageable. A and B are not, and then not fully utilized... That's a waste of IP addresses! \pause
    \end{itemize}
    Means to limit the number of nodes on a network (regardless of the class) and, thus, improve the manageability, are needed. Three means for it:
    \begin{itemize}
      \item Subnet, \pause
      \item VLSM (Variable Length Subnet Mask), \pause
      \item CIDR (Classless Inter-Domain Routing).
    \end{itemize}
  \end{frame}




  \begin{frame}
    \frametitle{Subnet and VLSM}
    \begin{itemize}
      \item Class A (16 m-addresses) and B (65 k-adresses) are too large! \pause
      \item Class C (254 addresses) is manageable. A and B are not, and then not fully utilized... That's a waste of IP addresses!
    \end{itemize}
  \end{frame}

  \begin{frame}
    \frametitle{Subnet and VLSM}
    \begin{figure}
        \centering
      \begin{tabular}{|r|cccc|}
        \hline
        Mask {\color{brown}/16} & {\color{brown}255} & {\color{brown}255} & {\color{fuchsia}0} & {\color{fuchsia}0} \\
        65.534 {\color{blue}hosts} & {\color{brown}11111111} & {\color{brown}11111111} & {\color{fuchsia}00000000} & {\color{fuchsia}00000000} \\ \hline
        \multirow{2}{*}{Network address} & \color{ForestGreen}172 & \color{ForestGreen}64 & \color{blue}0 & \color{blue}0 \\
        & \color{ForestGreen}10101100 & \color{ForestGreen}01000000 & \color{blue}00000000 & \color{blue}00000000 \\ \hline
        \multirow{2}{*}{First host} & \color{ForestGreen}172 & \color{ForestGreen}64 & \color{blue}0 & \color{blue}1 \\
        & \color{ForestGreen}10101100 & \color{ForestGreen}01000000 & \color{blue}00000000 & \color{blue}00000001 \\ \hline
        \multirow{2}{*}{Last host} & \color{ForestGreen}172 & \color{ForestGreen}64 & \color{blue}255 & \color{blue}254 \\
        & \color{ForestGreen}10101100 & \color{ForestGreen}01000000 & \color{blue}11111111 & \color{blue}11111110 \\ \hline
        \multirow{2}{*}{Broadcast address} & \color{ForestGreen}172 & \color{ForestGreen}64 & \color{blue}255 & \color{blue}255 \\
        & \color{ForestGreen}10101100 & \color{ForestGreen}01000000 & \color{blue}11111111 & \color{blue}11111111 \\ \hline
      \end{tabular}
      \caption{IP address example 2}
    \end{figure}
  \end{frame}

  \begin{frame}
    \frametitle{Subnet and VLSM}
    \begin{figure}
        \centering
      \begin{tabular}{|r|cccc|}
        \hline
        Mask {\color{brown}/12} & \color{brown}255 & \color{brown}24\color{fuchsia}0 & \color{fuchsia}0 & \color{fuchsia}0 \\
        1.048.574 {\color{blue}hosts} & \color{brown}11111111 & \color{brown}1111\color{fuchsia}0000 & \color{fuchsia}00000000 & \color{fuchsia}00000000 \\ \hline
        \multirow{2}{*}{Network address} & \color{ForestGreen}172 & \color{ForestGreen}6\color{blue}4 & \color{blue}0 & \color{blue}0 \\
        & \color{ForestGreen}10101100 & \color{ForestGreen}0100\color{blue}0000 & \color{blue}00000000 & \color{blue}00000000 \\ \hline
        \multirow{2}{*}{First host} & \color{ForestGreen}172 & \color{ForestGreen}6\color{blue}4 & \color{blue}0 & \color{blue}1 \\
        & \color{ForestGreen}10101100 & \color{ForestGreen}0100\color{blue}0000 & \color{blue}00000000 & \color{blue}00000001 \\ \hline
        \multirow{2}{*}{Last host} & \color{ForestGreen}172 & \color{ForestGreen}7\color{blue}9 & \color{blue}255 & \color{blue}254 \\
        & \color{ForestGreen}10101100 & \color{ForestGreen}0100\color{blue}1111 & \color{blue}11111111 & \color{blue}11111110 \\ \hline
        \multirow{2}{*}{Broadcast address} & \color{ForestGreen}172 & \color{ForestGreen}7\color{blue}9 & \color{blue}255 & \color{blue}255 \\
        & \color{ForestGreen}10101100 & \color{ForestGreen}0100\color{blue}1111 & \color{blue}11111111 & \color{blue}11111111 \\ \hline
      \end{tabular}
      \caption{IP address example 3}
    \end{figure}
  \end{frame}

  \begin{frame}
    \frametitle{Subnet and VLSM}
    \begin{figure}
        \centering
      \begin{tabular}{|r|cccc|}
        \hline
        Mask {\color{brown}/10} & \color{brown}255 & \color{brown}1\color{fuchsia}92 & \color{fuchsia}0 & \color{fuchsia}0 \\
         4.194.302 {\color{blue}hosts} & \color{brown}11111111 & \color{brown}11\color{fuchsia}000000 & \color{fuchsia}00000000 & \color{fuchsia}00000000 \\ \hline
        \multirow{2}{*}{Network address} & \color{ForestGreen}172 & \color{ForestGreen}6\color{blue}4 & \color{blue}0 & \color{blue}0 \\
        & \color{ForestGreen}10101100 & \color{ForestGreen}01\color{blue}000000 & \color{blue}00000000 & \color{blue}00000000 \\ \hline
        \multirow{2}{*}{First host} & \color{ForestGreen}172 & \color{ForestGreen}6\color{blue}4 & \color{blue}0 & \color{blue}1 \\
        & \color{ForestGreen}10101100 & \color{ForestGreen}01\color{blue}000000 & \color{blue}00000000 & \color{blue}00000001 \\ \hline
        \multirow{2}{*}{Last host} & \color{ForestGreen}172 & \color{ForestGreen}1\color{blue}27 & \color{blue}255 & \color{blue}254 \\
        & \color{ForestGreen}10101100 & \color{ForestGreen}01\color{blue}111111 & \color{blue}11111111 & \color{blue}11111110 \\ \hline
        \multirow{2}{*}{Broadcast address} & \color{ForestGreen}172 & \color{ForestGreen}1\color{blue}27 & \color{blue}255 & \color{blue}255 \\
        & \color{ForestGreen}10101100 & \color{ForestGreen}01\color{blue}111111 & \color{blue}11111111 & \color{blue}11111111 \\ \hline
      \end{tabular}
      \caption{IP address example 4}
    \end{figure}
  \end{frame}

  \begin{frame}
    \frametitle{Subnet and VLSM}
    \begin{figure}
        \centering
      \begin{tabular}{|r|cccc|}
        \hline
        Mask {\color{brown}/31} & \color{brown}255 & \color{brown}255 & \color{brown}255 & \color{fuchsia}254 \\
         0 {\color{blue}host} & \color{brown}11111111 & \color{brown}11111111 & \color{brown}11111111 & \color{brown}1111111\color{fuchsia}0 \\ \hline
        \multirow{2}{*}{Network address} & \color{ForestGreen}172 & \color{ForestGreen}64 & \color{ForestGreen}0 & \color{ForestGreen}25\color{blue}4 \\
        & \color{ForestGreen}10101100 & \color{ForestGreen}01000000 & \color{ForestGreen}00000000 & \color{ForestGreen}1111111\color{fuchsia}0 \\ \hline
        \multirow{2}{*}{First host} & \color{ForestGreen}172 & \color{ForestGreen}64 & \color{ForestGreen}0 & \color{blue}? \\
        & \color{ForestGreen}10101100 & \color{ForestGreen}01000000 & \color{ForestGreen}00000000 & \color{ForestGreen}1111111\color{fuchsia}? \\ \hline
        \multirow{2}{*}{Last host} & \color{ForestGreen}172 & \color{ForestGreen}64 & \color{ForestGreen}255 & \color{blue}? \\
        & \color{ForestGreen}10101100 & \color{ForestGreen}01000000 & \color{ForestGreen}00000000 & \color{ForestGreen}1111111\color{fuchsia}? \\ \hline
        \multirow{2}{*}{Broadcast address} & \color{ForestGreen}172 & \color{ForestGreen}64 & \color{ForestGreen}255 & \color{blue}255 \\
        & \color{ForestGreen}10101100 & \color{ForestGreen}01000000 & \color{ForestGreen}00000000 & \color{ForestGreen}1111111\color{fuchsia}1 \\ \hline
      \end{tabular}
      \caption{IP address example 5}
    \end{figure}
  \end{frame}


  \begin{frame}
    \frametitle{Subnet and VLSM}
    \begin{figure}
        \centering
      \begin{tabular}{|r|cccc|}
        \hline
        Mask {\color{brown}/30} & \color{brown}255 & \color{brown}255 & \color{brown}255 & \color{fuchsia}252 \\
         2 {\color{blue}hosts} & \color{brown}11111111 & \color{brown}11111111 & \color{brown}11111111 & \color{brown}111111\color{fuchsia}00 \\ \hline
        \multirow{2}{*}{Network address} & \color{ForestGreen}172 & \color{ForestGreen}64 & \color{ForestGreen}0 & \color{ForestGreen}25\color{blue}2 \\
        & \color{ForestGreen}10101100 & \color{ForestGreen}01000000 & \color{ForestGreen}00000000 & \color{ForestGreen}1111111\color{fuchsia}00 \\ \hline
        \multirow{2}{*}{First host} & \color{ForestGreen}172 & \color{ForestGreen}64 & \color{ForestGreen}0 & \color{ForestGreen}25\color{blue}3 \\
        & \color{ForestGreen}10101100 & \color{ForestGreen}01000000 & \color{ForestGreen}00000000 & \color{ForestGreen}1111111\color{fuchsia}01 \\ \hline
        \multirow{2}{*}{Last host} & \color{ForestGreen}172 & \color{ForestGreen}64 & \color{ForestGreen}255 & \color{ForestGreen}25\color{blue}4 \\
        & \color{ForestGreen}10101100 & \color{ForestGreen}01000000 & \color{ForestGreen}00000000 & \color{ForestGreen}1111111\color{fuchsia}10 \\ \hline
        \multirow{2}{*}{Broadcast address} & \color{ForestGreen}172 & \color{ForestGreen}64 & \color{ForestGreen}255 & \color{ForestGreen}25\color{blue}5 \\
        & \color{ForestGreen}10101100 & \color{ForestGreen}01000000 & \color{ForestGreen}00000000 & \color{ForestGreen}1111111\color{fuchsia}11 \\ \hline
      \end{tabular}
      \caption{IP address example 6}
    \end{figure}
  \end{frame}



  \begin{frame}
    %\frametitle{Subnet masks cheat sheet}
    \begin{figure}
      \centering
      \resizebox{7cm}{!} {
	\begin{tabular}{|lccr|}
	  \hline
	  \multicolumn{2}{|c|}{Netmask} & CIDR & hosts \\ \hline
	  255.255.255.255 & 11111111.11111111.11111111.11111111 & /32 & single address \\
	  255.255.255.254 & 11111111.11111111.11111111.11111110 & /31 & Unusable \\
	  255.255.255.252 & 11111111.11111111.11111111.11111100 & /30 & 2 \\
	  255.255.255.248 & 11111111.11111111.11111111.11111000 & /29 & 6 \\
	  255.255.255.240 & 11111111.11111111.11111111.11110000 & /28 & 14 \\
	  255.255.255.224 & 11111111.11111111.11111111.11100000 & /27 & 30 \\
	  255.255.255.192 & 11111111.11111111.11111111.11000000 & /26 & 62 \\
	  255.255.255.128 & 11111111.11111111.11111111.10000000 & /25 & 126 \\
	  255.255.255.0   & 11111111.11111111.11111111.00000000 & /24 & 254 \\
	  255.255.254.0   & 11111111.11111111.11111110.00000000 & /23 & 510 \\
	  255.255.252.0   & 11111111.11111111.11111100.00000000 & /22 & 1.022 \\
	  255.255.248.0   & 11111111.11111111.11111000.00000000 & /21 & 2.046 \\
	  255.255.240.0   & 11111111.11111111.11110000.00000000 & /20 & 4.094 \\
	  255.255.224.0   & 11111111.11111111.11100000.00000000 & /19 & 8.190 \\
	  255.255.192.0   & 11111111.11111111.11000000.00000000 & /18 & 16.382 \\
	  255.255.128.0   & 11111111.11111111.10000000.00000000 & /17 & 32.766 \\
	  255.255.0.0     & 11111111.11111111.00000000.00000000 & /16 & 65.534 \\
	  255.254.0.0     & 11111111.11111110.00000000.00000000 & /15 & 131.070 \\
	  255.252.0.0     & 11111111.11111100.00000000.00000000 & /14 & 262.142 \\
	  255.248.0.0     & 11111111.11111000.00000000.00000000 & /13 & 524.286 \\
	  255.240.0.0     & 11111111.11110000.00000000.00000000 & /12 & 1.048.574 \\
	  255.224.0.0     & 11111111.11100000.00000000.00000000 & /11 & 2.097.152 \\
	  255.192.0.0     & 11111111.11000000.00000000.00000000 & /10 & 4.194.302 \\
	  255.128.0.0     & 11111111.10000000.00000000.00000000 & /9  & 8.388.606 \\
	  255.0.0.0       & 11111111.00000000.00000000.00000000 & /8  & 16.777.214 \\
	  254.0.0.0       & 11111110.00000000.00000000.00000000 & /7  & 33.554.430\\
	  252.0.0.0       & 11111100.00000000.00000000.00000000 & /6  & 67.108.862\\
	  248.0.0.0       & 11111000.00000000.00000000.00000000 & /5  & 134.217.726\\
	  240.0.0.0       & 11110000.00000000.00000000.00000000 & /4  & 268.435.454\\
	  224.0.0.0       & 11100000.00000000.00000000.00000000 & /3  & 536.870.910\\
	  192.0.0.0       & 11000000.00000000.00000000.00000000 & /2  & 1.073.741.822\\
	  128.0.0.0       & 10000000.00000000.00000000.00000000 & /1  & 2.147.483.646\\
	  0.0.0.0         & 00000000.00000000.00000000.00000000 & /0  & IP space \\ \hline
	\end{tabular}
      }
      \caption{Subnet mask cheat sheet}
    \end{figure}
  \end{frame}


  \begin{frame}
    \frametitle{Subnet and VLSM}
    That's it for today's. Let's practice before starting routing!
  \end{frame}
