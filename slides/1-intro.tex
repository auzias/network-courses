\section{Introduction}
\subsection{Definitions and presentation}
  \begin{frame}
    \frametitle{Definitions}
      \begin{itemize}
        \item \textbf{Network:} an \textbf{interconnected} group or system\pause
        \item \textbf{Internet:} world wide \textbf{interconnected system of network\emph{s}} \color{blue}\href{http://tools.ietf.org/html/rfc791}{RFC791 (1981)}\color{black}\pause
        \item \textbf{IP:} Internet \textbf{Protocol} that provides the functions necessary to deliver a package of bits from a source to a destination over a network\pause
        \item \textbf{(world wide) Web:} \textbf{network} consisting of a collection of Internet websites using HTTP\pause
        \item \textbf{HTTP:} Hypertext Transfer Protocol \textbf{Protocol}, application-level protocol for distributed, collaborative, hypermedia information systems \color{blue}\href{http://tools.ietf.org/html/draft-ietf-httpbis-http2-14}{draft HTTP2 (July 2014)} \color{black}
      \end{itemize}
  \end{frame}
  \begin{frame}
    \frametitle{Definitions}
      \begin{itemize}
        \item \textbf{Router:} network \textbf{hardware} providing routing services\pause
        \item \textbf{Routing:} \textbf{algorithm processed} to decide where to forward a packet\pause
        \item \textbf{Forwarding:} \textbf{\emph{action}} of moving a packet from an NIC to another\pause
        \item \textbf{NIC:} Network Interface Card
        \item \textbf{Switch (hub):} network \textbf{hardware} that connect systems together using packet switching\pause
        \item \textbf{Packet switching:} forward-like method regardless of the content (destination-based)
      \end{itemize}
  \end{frame}
  \begin{frame}
    \frametitle{Definitions}
      \begin{itemize}
        \item \textbf{Node (network):} any entity that can send/receive packets from a network through a NIC\pause
        \item \textbf{Client:} \textbf{computer} able to send requests to a server\pause
        \item \textbf{Request:} \textbf{application message} destined for a server (\emph{order})\pause
        \item \textbf{Server:} \textbf{computer} able to respond a client's requests\pause
        \item \textbf{Request:} \textbf{application message} destined for a client (\emph{result})\pause
        \item \textbf{Fat client:} \textbf{application} where most functions are processed by the client itself\pause
        \item \textbf{Thin client:} \textbf{application} where most functions are carried out on a central server
      \end{itemize}
  \end{frame}

\subsection{Network classification}
  \begin{frame}
    \frametitle{What kind of networks is it ?}
      \begin{itemize}
        \item \textbf{BAN:} Body Area Network\pause
        \item \textbf{PAN:} Personal Area Networks\pause
        \item \textbf{(W)LAN:} (Wireless) Local Area Networks (home, office, school or airport)\pause
        \item \textbf{MAN:} Metropolitan Area Networks, can cover a whole city\pause
        \item \textbf{WAN:} Wide Area Networks cover a broad area (Internet)
      \end{itemize}
  \end{frame}
  \begin{frame}
    \frametitle{Topologies}
    \begin{figure}[t]
      \centering
      \includegraphics[height=5cm]{./imgs/topologies.png}
      \caption{\color{blue}\href{https://upload.wikimedia.org/wikipedia/commons/thumb/9/97/NetworkTopologies.svg/640px-NetworkTopologies.svg.png}{upload.wikimedia.org}}
      \label{fig:ntwks}
    \end{figure}
      \end{frame}
  \begin{frame}
    \frametitle{Topologies}
    \begin{itemize}
      \item \textbf{Point-to-point:} two entities directly connected to each other (tunnel).\pause
      \item \textbf{Ring:} data go around the ring, unidirectional way network.\pause
      \item \textbf{Mesh:} all nodes cooperate in the distribution of data in the network\footnote{\color{blue}\href{http://www.newscientist.com/article/dn26285-hong-kong-protesters-use-a-mesh-network-to-organise.html}{Hong Kong protesters use a mesh network to organize}}.\pause
      \item \textbf{Star:} all messages go through the same central node, reducing network failure.\pause
      \item \textbf{Fully connected:} all nodes are connected to all other nodes.\pause
      \item \textbf{Line:} bidirectional link between two nodes. Node can only send packet going through its neighbors.\pause
      \item \textbf{Bus:} all nodes are connected to the same media. Only one at a time can send packet, that all other receives.\pause
      \item \textbf{Tree:} hierarchical topology, such as, i.e., binary tree.
    \end{itemize}
  \end{frame}
  \begin{frame}
    \frametitle{Bonus}

      \end{frame}


\subsection{HTTP request/response example}
\begin{frame}
    \frametitle{HTTP request/response example}
      Enter \color{blue}\href{http://getbootstrap.com}{getbootstrap.com} \color{black} in your browser\pause
      \begin{figure}
    \includegraphics[width=11.5cm]{./imgs/dns-req.png}
  \caption{DNS request/response}
      \end{figure}
      \pause
      \begin{figure}
    \includegraphics[trim = 0 0 100mm 0, clip, width=11.5cm]{./imgs/http-req.png}
  \caption{HTTP request/response}
      \end{figure}
  \end{frame}
    \begin{frame}
    \frametitle{How does messages reach destination ?}
      \begin{figure}
    \includegraphics[width=9.5cm]{./imgs/routing.jpg}
  \caption{\color{blue}\href{http://acenk90.files.wordpress.com}{acenk90.files.wordpress.com}}
  \label{fig:routing}
      \end{figure}
  \end{frame}
    \begin{frame}
    \frametitle{More like this...}
      \begin{figure}
    \includegraphics[height=6.5cm]{./imgs/map.jpg}
  \caption{\color{blue}\href{https://upload.wikimedia.org/wikipedia/commons/thumb/d/d2/Internet_map_1024.jpg/768px-Internet_map_1024.jpg}{wikimedia.org}}
  \label{fig:routing}
      \end{figure}
  \end{frame}

\subsection{Models overview (OSI and TCP/IP)}
  \begin{frame}
    \frametitle{How does it work ? From signal to application...}
    \begin{figure}[t]
      \centering
      \includegraphics[height=6cm]{./imgs/layers.png}
      \caption{\color{blue}\href{http://mycomsats.com/blogs/wp-content/uploads/2012/05/image.png}{mycomsats.com}}
      \label{fig:ntwks}
    \end{figure}
  \end{frame}