\documentclass[11pt]{article}
\usepackage[utf8x]{inputenc}
\usepackage[english]{babel}
\usepackage{graphicx}
\usepackage{wrapfig}
\usepackage[margin=2cm]{geometry}
\usepackage{color}
\usepackage{hyperref}

\begin{document}
 \bibliographystyle{plain}
\title{Project guidelines and marking criteria}
\date{Fall 2015}
\author{Maël Auzias}
\maketitle

\section{Introduction}
The project for this network course will be part of your assessment. Each group (2 students per project) must give a presentation on the third week of September. Each presentation will last 15 min (11 min presentation + 4 min questions). The aim of this project is to practice and learn how a particular aspect of a network works. This means that, through this project, you need to learn how the lower layers interact!\\
\textbf{During this project, do not waste any time on the GUI! Spend time on understanding how client and server interact by capturing and analyzing packets.}

\section{Overview (2015-09-04)}
Beforehand, each group will need to give a short overview defining:
  \begin{itemize}
    \item the name of the project,
    \item the objectives of the project,
    \item what you want to learn with this project,
    \item how you will organize yourselves to share the code (USB key, email, git, svn...).
  \end{itemize}
The overview must be sent by email (To: teaching@auzias.net, Cc:your-partner(s)@email.mail), with the subject: \textbf{[overview] student1's name - student2's name}

\section{Report and Presentation (date to come)}
The report and the presentation (\textbf{both} pdf files) must be sent by email (To: teaching@auzias.net, Cc:your-partner@email.mail), with the subject: \textbf{[report] student1's name - student2's name}.
\subsection{Report}
A 5 to 10 pages-pdf report will be handed in, explaining:
  \begin{itemize}
    \item the name of the project,
    \item the objectives of the project and the differences, if any, with the Overview's Objectives,
    \item the explanation on why you did not meet your objectives and what are the possible solutions you had thought of to meet them,
    \item what you wanted to learn with this project and what you really learned,
    \item how you did share the code,
    \item between 3 and 6 specific captured packets and their explanations,
    \item what are the challenges encountered and how you overcome them (this part can be a \emph{subsubsection} of what you really learned),
    \item what you did to {\bf legally} test security flaw.
  \end{itemize}
If the report is made with \LaTeX, extra credit will be given.

\subsection{Presentation}
A 10 minute presentation (11 min presentation + 4 min questions) to explain your project, especially what you learned (and how the lower layers' nodes interact).

If the presentation is made with beamer, extra credit will be given.

\end{document}
