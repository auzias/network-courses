\documentclass[11pt]{article}
\usepackage[utf8x]{inputenc}
\usepackage[english]{babel}
\usepackage{graphicx}
\usepackage{wrapfig}
\usepackage[margin=2cm, tmargin=0cm]{geometry}
\usepackage{color}
\usepackage{hyperref}

\begin{document}
 \bibliographystyle{plain}
\title{Network project}
\date{Fall 2014}
\author{Maël Auzias}
\maketitle

\section{Introduction}
The project for this network course will form part of your assessment. Each group (2 students per project) must give a presentation from the 10\textsuperscript{th} December. Each presentation will last 10 min (6 min presentation + 4 min questions).
The aim of this project is to practice and learn how a particular aspect of a network works. This means that, through this project, you need to learn how the lower layers interact!

\section{Overview (2014-10-17)}
Beforehand, each group will need to give a short overview defining:
  \begin{itemize}
    \item the name of the project,
    \item the objectives of the project,
    \item what you want to learn with this project,
    \item how you will organize yourselves to share the code (USB key, email, git, svn...).
  \end{itemize}

\section{Report and Presentation (2014-12-10)}
The report and the presentation (\textbf{both} pdf files) must be sent by email (To: teaching@auzias.net, Cc:your-partner@email.mail), with the subject: \textbf{[NETWORK] student1's name - student2's name}, before the 10\textsuperscript{th} December 2014 08:00 AM.
\subsection{Report}
A 5-page-pdf report will be handed in containing:
  \begin{itemize}
    \item the name of the project,
    \item the objectives of the project (and the differences, if any, with the Overview's Objectives),
    \item what you wanted to learn with this project and what you really learned,
    \item how you did share the code,
    \item between 3 and 6 specific captured packets and their explanations,
    \item the challenges encountered and how you overcame them (this part can be a \emph{subsubsection} of what you really learned).
  \end{itemize}
If the report is made with \LaTeX, extra credit will be given.

\subsection{Presentation}
A 15 minute presentation to promote your project, especially what you learned (and how the lower layers' nodes interact).

If the presentation is made with beamer, extra credit will be given.

\end{document}
