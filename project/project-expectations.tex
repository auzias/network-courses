\documentclass[11pt]{article}
\usepackage[utf8x]{inputenc}
\usepackage[english]{babel}
\usepackage{graphicx}
\usepackage{wrapfig}
\usepackage[margin=3cm, tmargin=2cm]{geometry}
\usepackage{color}
\usepackage{hyperref}

\begin{document}
 \bibliographystyle{plain}
\title{Network project}
\date{Fall 2014}
\author{Maël Auzias}
\maketitle

\section{Introduction}
The project of this network course will be part of the grading system. Every group (2 students per project) must perform a presentation the 2014-12-17. Each presentation will last 10 min (6 min presentation + 4 min questions).
The aim of this project is to practice and learn how do a peculiar aspect of networks work. Meaning that you need to learn thanks to this project how do the lower layers interact!

\section{Student production}
\subsection{Overview (2014-10-17)}
Beforehand, each group will need to give a short overview defining:
  \begin{itemize}
    \item name of the project,
    \item objectives of the project,
    \item what you want to learn with this project,
    \item how you will organize themself to share the code (USB key, email, git, svn...).
  \end{itemize}
\subsection{Report (2014-12-10)}
A 5-pages report will be handed explaning:
  \begin{itemize}
    \item name of the project,
    \item objectives of the project,
    \item what you want to learn with this project,
    \item how you really organize,
    \item what are the challenges encountered and how you overpass them.
  \end{itemize}
\subsection{Presentation (2014-12-17)}
A 10 minutes presentation. No rule, just promote your project, especially what you learn (and how interact the nodes with the lower layers).


\end{document}
