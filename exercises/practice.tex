\documentclass[11pt]{article}
\usepackage[utf8x]{inputenc}
\usepackage[english]{babel}
\usepackage{graphicx}
\usepackage{wrapfig}
\usepackage[margin=3cm, tmargin=2cm]{geometry}
\usepackage{color}
\usepackage{hyperref}

\begin{document}
 \bibliographystyle{plain}%Choose a bibliograhpic style
\title{Network training}
\date{Fall 2014}
\author{Maël Auzias}
\maketitle

\tableofcontents
\pagebreak


\section{HTTP example}
\subsection{Who are you? Where are you?}
What is your own IP address? What is your own MAC address? What is your network mask?
What do theses commands display?
\begin{verbatim}
#ifconfig
$curl ifconfig.me
$netstat -at
\end{verbatim}

\subsubsection{How to get out?}
Before we can access the Internet we need to know who/what is the gateway. What is a gateway?
What do theses commands display?
\begin{verbatim}
#route -n
#arp -a
\end{verbatim}

\subsubsection{What's your number?}
As explained before, humans can easily remember name such as \color{blue}\href{https://news.ycombinator.com}{news.ycombinator.com}\color{black} or \color{blue}\href{http://root-me.org}{root-me.org}\color{black} but it is not as easy to remind \verb"198.41.191.47" or \verb"212.129.28.16". We need a way to \emph{translate} a domain name into an IP address. This is role of DNS\footnote{DNS: Domain Name Server, if you needed to read this footnote keep in mind that you should remember it from now on}. You can query DNS using \verb"nslookup".

\subsubsection{Wait! What direction?}
The (IP) address is of the website you want to visit is now known. The next step is to know how to \emph{GET}\footnote{that's not a HTTP verb, but a word play!} there. Try to trace the route using... \verb"traceroute" (or \verb"tracepath") to see packets' path. Do you know any town on the path from where you are to \verb"www.ethicalhacker.net" server?
Note that sometimes, for security reasons, IMCP protocol is blocked. If this is the case you can use an option to use TCP SYN for probes.
How does \verb"traceroute" work ?

\subsubsection{Go GET it!}
What does \verb"wget 95.215.16.43 80" do?

\subsubsection{Capture it}
Use \verb"wireshark" to capture:
\begin{itemize}
  \item a GET through HTTP (\color{blue}\href{http://selfoss.aditu.de/}{selfoss.aditu.de}\color{black} does not have valid HTTPS certificate).
  \item a GET through HTTPS (\color{blue}\href{https://micahflee.com/}{micahflee.com}\color{black} force redirection to HTTPS).
\end{itemize}
What differences can you see? How can you explain theses differences?

\subsubsection{"Security" \emph{without} HTTPS}
Some methods allow web-master to secure some part of the website. Then the website requires a user and a password to enter. You can test on the webpage: http://teaching.auzias.net/http-auth/
\begin{itemize}
  \item user: test
  \item pass: p4ssw0rd
\end{itemize}
Use \verb"wireshark" and verify if you had captured the user:password encrypted or not. \color{blue}\href{http://tools.ietf.org/html/rfc2617}{RFC 2617}\color{black} could be handful.

\section{Chat}
\subsection{netcat}
\verb"netcat (or ncat)" is a "network swiss army knife". By checking its man page how can you use it as a chat server/client (two nodes only).
\subsubsection{TCP}
Use the mode TCP of \verb"netcat" and try it. Can \verb"netstat" could, somehow, be helpful for anything while waiting for connection? Can \verb"telnet" be used to chat?
\subsubsection{UDP}\label{chat:udp}
Use the mode UDP of \verb"netcat" and try it. Explain a situation within the server could not receive every packet.
\newline
More example of netcat:\color{blue}\href{http://www.brianhaddock.com/2013/8-great-netcat-nc-unix-commands}{brianhaddock.com}\color{black}

\subsection{Create your own client}
In the two next exercise you can use \verb"netcat" to verify, whether or not, your client are working.
\subsubsection{TCP}
Create a TCP client that send packet to a specific port on \verb"localhost". (The class \verb"Socket" should be useful...)
\subsubsection{UDP}
Try to produce the situation explained in \ref{chat:udp} by implementing it using Java language and flood a specific port on \verb"localhost". (The classes \verb"InetAddress" and \verb"DatagramPacket" should be useful...)

\subsection{Create your own server}
In the two next exercise you can use your previous client to verify, whether or not, your server are working.
\subsubsection{TCP}
Implement in Java a TCP server that connects at the first attempt of connection and displays it on the screen. Next step is to \emph{echo} it (send it back).
\subsubsection{UDP}
Implement in Java a UDP server that receives and displays it on the screen. Next step is to \emph{echo} it (send it back).

\end{document}
